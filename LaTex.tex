\documentclass[twocolumn, a4j]{article}
\usepackage{multirow}
\usepackage{amsmath,amssymb}
\usepackage[T1]{fontenc}

\title{NECO 課題}
\renewcommand{\thefootnote}{\fnsymbol{footnote}}
\author{Kosuke Watanabe\footnotemark[2] }
\renewcommand{\thefootnote}{\arabic{footnote}}
\date{\today}

\begin{document}

\twocolumn[
\begin{@twocolumnfalse}
  \maketitle
  \vspace{-6mm}
  \begin{abstract}
    判定システム”ホークアイ”に関する論文を読み、まとめた。
  \end{abstract}
  \vspace{2mm}
\end{@twocolumnfalse}
]

\renewcommand{\thefootnote}{\fnsymbol{footnote}}
\footnotetext[2]{慶應義塾大学 村井研}
\renewcommand{\thefootnote}{\arabic{footnote}}

\section{はじめに}

 ホークアイを中心に,現代のスポーツ周りに取り憑いたテクノロジーの利用へと傾く欲 望に焦点を合わせ,そのありようについて考察する。

\section{先行研究}
判定補助テクノロジーの利用については,種々の分野で学術的な考察が重ねられてきている。 そして,その中でもっとも早い時期のものは法学分野のものであった。
なぜなら、判定補助という制度は“一度出た判決=判定に対する再審理が行われること”になるため、法学的な議論が行われるためである。

\section{誤審という始まり}
誤審は負けたチームのサポーターたちの不満のはけ口となる口実から始まった。
その後、テレビ中継によりリプレイが開始されたことにより人々はさまざまな場面を繰り返し見ることができるようになった。
そしてカメラ機能の向上とホークアイの誕生により、ほぼリアルタイムでプレイを確認できるようになった。

\section{映像としてのホークアイ}
テニスにおけるホークアイの誤差は平均 2.6 mm であるという。これはボールの外側の繊維のせいでカメラがボールの輪郭を正確に定義できないためである。
コートの環境等の諸条件によって誤差の範囲は変化するが,いずれにしてもホークアイ の判定には「不確実性のゾーン」と言うべき範囲が必ず存在する。
しかしルールブックにはホークアイの判定は最終的なものであり、抗議することができないと書かれている。
このような不確実な判定をするホークアイにわれわれは審判の存在論的権威を譲り渡しているためこの点を改善しなければならない。

\section{結び}

ホークアイの映像には判定に余分な動画部分があり、それは大衆的な 視覚文化の引用に満ちたものであった。そしてホークアイには不確実性のゾーンというものがある。しかし、だからといってホークアイのような判定補助テクノロジーがスポーツの場に居場所を持つべきではないというわけではない。不確実性のゾーンに対して謙虚であるならば,つまり、絶対的判定者として振る舞わないのであれば、その有用性によってこうしたテクノロジーには居場所が与えられるべきである。ただし,その場所は審判の存在論的権威の内側,つまり、審判の手の中になければならないのであり、そのような形で運用が今 後検討されるべきである。

\renewcommand{\refname}{参考文献}
\begin{thebibliography}{数字}
  \bibitem[opt]{key} スポーツとテクノロジー:ホークアイシステムの場合(柏原 全孝)
\end{thebibliography}

\end{document}
